\documentclass[10pt,a4paper]{article}
\usepackage{blindtext}
\usepackage{subcaption}
\usepackage{graphicx}
\usepackage{tikz}
\usepackage{amssymb}
\usepackage{caption}
\usepackage{amsmath}
\usepackage{circuitikz}
\usepackage{hyperref}
\usepackage{amssymb}
\usepackage{amsmath}
\usepackage{listings}

\lstset{
    inputencoding=utf8,
    extendedchars=true,
    literate={á}{{\'a}}1 {é}{{\'e}}1 {í}{{\'i}}1 {ó}{{\'o}}1 {ú}{{\'u}}1 {ñ}{{\~n}}1 {Á}{{\'A}}1 {É}{{\'E}}1 {Í}{{\'I}}1 {Ó}{{\'O}}1 {Ú}{{\'U}}1 {Ñ}{{\~N}}1
}
\input{AEDmacros}
\newcommand{\notimplies}{\;\not\!\!\!\implies}
\title{Algoritmos y Estructuras de Datos III}
\author{Tomás Agustín Hernández}
\date{}

\begin{document}
\maketitle

\begin{figure}[b]
    \centering
    \begin{tikzpicture}[remember picture,overlay]
        \node[anchor=south east, inner sep=0pt, xshift=-1cm, yshift=2cm] at (current page.south east) {
            \begin{minipage}[b]{0.5\textwidth}
                \includegraphics[width=\linewidth]{logo_uba.jpg}
                \label{fig:bottom}
            \end{minipage}
        };
    \end{tikzpicture}
\end{figure}

\newpage
\section*{Grafos}
Un grafo es una estructura de datos compuesta por \textbf{V un conjunto de vértices (nodos)} y \textbf{E un conjunto de aristas que conectan pares de vértices}. \\
Definimos un grafo como: $G = (V, E)$ donde $E \subseteq V X V$.
\[\begin{minipage}[b]{0.7\textwidth}
    \includegraphics[width=\linewidth]{assets/graph.png}
\end{minipage}\]
Nótese que tiene sentido que el conjunto de aristas E esté formado por un par $V X V$. De lo contrario no existiría una arista.
\subsection*{Cantidad de vertices y cantidad de aristas}
Definimos la cantidad de vértices de un grafo como $\longitud{V} = n$ y la cantidad de aristas como $\longitud{E} = m$. \\
\textbf{Good to Know}: En un grafo T (árbol) \textbf{siempre} sucede que $\longitud{V} > \longitud{E}$ en una unidad. Lo veremos más adelante, pero está bueno notarlo.
\subsection*{Grafo Simple}
Un Grafo Simple es un grafo que no tiene más de una arista entre dos mismos nodos. Nótese que aquí no importa la relación entre ellos, sino que, no debería estar repetido el mismo par (sin importar el orden).
\begin{itemize}
    \item $E = \{(A,B), (A,B)\}$ NO es un Grafo Simple.
    \item $E = \{(A,B), (B,A)\}$ o $E = \{(B,A), (A,B)\}$ NO es un Grafo Simple.
    \item $E = \{(A,B)\}$ SÍ es un Grafo Simple.
\end{itemize}
\subsection*{Grafo no Simple}
Un Grafo no Simple es un grafo que no tiene ninguna restricción con respecto a la relación entre dos vértices (nodos). 
\begin{itemize}
    \item $E = \{(A,B), (A,B)\}$ es un Grafo no Simple.
    \item $E = \{(A,B), (B,A)\}$ o $E = \{(B,A), (A,B)\}$ es un Grafo no Simple.
    \item $E = \{(A,B)\}$ SÍ es un Grafo no Simple.
\end{itemize}
\textbf{Good to Know}: Los grafos simples, son subconjuntos de grafos no simples. Por ende, si tenemos un grafo no simple que cumple las propiedades de un grafo simple, consideramos que es un \textbf{grafo simple} al ser más restrictivo. \\
En los ejemplos anteriores, entonces, $E = \{(A,B)\}$ sería mejor considerado como Grafo Simple.
\end{document} 

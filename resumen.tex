\documentclass[10pt,a4paper]{article}
\usepackage{blindtext}
\usepackage{subcaption}
\usepackage{graphicx}
\usepackage{tikz}
\usepackage{amssymb}
\usepackage{caption}
\usepackage{amsmath}
\usepackage{circuitikz}
\usepackage{hyperref}
\usepackage{amssymb}
\usepackage{amsmath}
\usepackage{listings}

\lstset{
    inputencoding=utf8,
    extendedchars=true,
    literate={á}{{\'a}}1 {é}{{\'e}}1 {í}{{\'i}}1 {ó}{{\'o}}1 {ú}{{\'u}}1 {ñ}{{\~n}}1 {Á}{{\'A}}1 {É}{{\'E}}1 {Í}{{\'I}}1 {Ó}{{\'O}}1 {Ú}{{\'U}}1 {Ñ}{{\~N}}1
}
\input{AEDmacros}
\newcommand{\notimplies}{\;\not\!\!\!\implies}
\title{Algoritmos y Estructuras de Datos III}
\author{Tomás Agustín Hernández}
\date{}

\begin{document}
\maketitle

\begin{figure}[b]
    \centering
    \begin{tikzpicture}[remember picture,overlay]
        \node[anchor=south east, inner sep=0pt, xshift=-1cm, yshift=2cm] at (current page.south east) {
            \begin{minipage}[b]{0.5\textwidth}
                \includegraphics[width=\linewidth]{logo_uba.jpg}
                \label{fig:bottom}
            \end{minipage}
        };
    \end{tikzpicture}
\end{figure}

\newpage
\section*{Grafos}
Un grafo es una estructura de datos compuesta por \textbf{V un conjunto de vértices (nodos)} y \textbf{E un conjunto de aristas que conectan pares de vértices}. \\
Definimos un grafo como: $G = (V, E)$ donde $E \subseteq V X V$.
\[\begin{minipage}[b]{0.7\textwidth}
    \includegraphics[width=\linewidth]{assets/graph.png}
\end{minipage}\]
Nótese que tiene sentido que el conjunto de aristas E esté formado por un par $V X V$. De lo contrario no existiría una arista.
\subsection*{Cantidad de vertices y cantidad de aristas}
Definimos la cantidad de vértices de un grafo como $\longitud{V} = n$ y la cantidad de aristas como $\longitud{E} = m$. \\
\textbf{Good to Know}: En un grafo T (árbol) \textbf{siempre} sucede que $\longitud{V} > \longitud{E}$ en una unidad. Lo veremos más adelante, pero está bueno notarlo.
\subsection*{Grafo Simple}
Un Grafo Simple es un grafo que no tiene más de una arista entre dos mismos nodos. Nótese que aquí no importa la relación entre ellos, sino que, no debería estar repetido el mismo par (sin importar el orden).
\begin{itemize}
    \item $E = \{(A,B), (A,B)\}$ NO es un Grafo Simple.
    \item $E = \{(A,B), (B,A)\}$ o $E = \{(B,A), (A,B)\}$ NO es un Grafo Simple.
    \item $E = \{(A,B)\}$ SÍ es un Grafo Simple.
\end{itemize}
\subsection*{Grafo no Simple}
Un Grafo no Simple es un grafo que no tiene ninguna restricción con respecto a la relación entre dos vértices (nodos). 
\begin{itemize}
    \item $E = \{(A,B), (A,B)\}$ es un Grafo no Simple.
    \item $E = \{(A,B), (B,A)\}$ o $E = \{(B,A), (A,B)\}$ es un Grafo no Simple.
    \item $E = \{(A,B)\}$ SÍ es un Grafo no Simple.
\end{itemize}
\textbf{Good to Know}: Los grafos simples, son subconjuntos de grafos no simples. Por ende, si tenemos un grafo no simple que cumple las propiedades de un grafo simple, consideramos que es un \textbf{grafo simple} al ser más restrictivo. \\
En los ejemplos anteriores, entonces, $E = \{(A,B)\}$ sería mejor considerado como Grafo Simple.
\subsection*{Grado de un vértice (nodo)}
El grado de un vértice no es más que la cantidad de conexiones que tiene un nodo. \\
Definimos formalmente al grado de un vértice como $deg(v) = \#\{e \in E : v \in e\}$ \\
\textbf{Good to Know}: deg refiere a degree. \\
\textbf{Good to Know 2}: \textbf{e} es una arista particular $e = (v, w)$.
\subsection*{Grafo Dirigido}
Un Grafo Dirigido es un tipo de grafo (que puede ser simple o no simple) pero importa quién se relaciona con quién, es decir, el orden en que guardamos la relación. En dibujos, lo notamos con una flecha. 
\begin{itemize}
    \item $E = \{(A,B)_{\rightarrow}, (B, A)_{\leftarrow} \}$ es un grafo dirigido, esto quiere decir que $(A,B) \neq (B, A)$
\end{itemize}
\textbf{Good to Know}: A nivel estructura de datos, parece que E es igual en un Grafo Dirigido que un Grafo No Dirigido. No obstante, optamos, a nivel de notación incluir una fecha para indicar la relación entre ambos. A nivel código, deberías manejar alguna información extra para entender si es dirigido o no.
\subsubsection*{Grado de un vértice en un Grafo Dirigido}
En un Grafo Dirigido, el grado de un vértice se calcula como $deg(v) = indeg(v) + outdeg(v)$ \\
¿A qué nos referimos con indeg(v) y outdeg(v)? Como es un grafo dirigido, la relación entre los vértices importa. Por lo tanto, aquellas aristas entrantes al nodo v origen las agrupamos en indeg(v) mientras que aquellas aristas salientes desde el nodo v las agrupamos en outdeg(v).
\begin{itemize}
    \item $E = \{(A,B)_{\rightarrow}, (B, A)_{\leftarrow}, (A, C)_{\rightarrow} \}$ es un grafo dirigido. Siendo indeg(A) = 1, outdeg(A) = 2 $\implies deg(A) = 3$ 
\end{itemize}
\subsection*{Grafo No Dirigido}
Un Grafo No Dirigido es un grafo el cual no existe la dirección en una relación entre dos vértices.
\begin{itemize}
    \item $E = \{(A,B), (B, A) \}$ es un grafo no dirigido, esto quiere decir que $(A,B) = (B, A)$
\end{itemize}
\textbf{Good to Know}: Los Grafos Dirigidos o Grafos no Dirigidos pueden ser Grafos Simples o no Simples.
\subsubsection*{Cantidad de Aristas de un Grafo Simple no Dirigido}
Como no importa la relación entre los vértices pues $(A,B) = (B,A)$, contamos solo una vez esa arista \[0 \leq |E| \leq \frac{n(n-1)}{2} = \binom{n}{2}\]
\subsubsection*{Grado de un vértice en un Grafo No Dirigido}
Es la misma cuenta que hacemos en un grafo dirigido, pero sin distinguir en grupos a las aristas \textbf{(porque no tienen dirección)}.
\subsection*{Grafo Completo}
Llamamos Grafo Completo a un Grafo que tiene todos sus nodos conectados.
\subsection*{Camino}
Un camino en un grafo es una sucesión de aristas. \\
Lo definimos formalmente de la siguiente manera: $e_{i}^{f} = e_{i+1}^{0} \ \forall i \ (v = e^{0}_{i}, w = e^{f}_{r})$ \\
\textbf{Good to Know}: El $0$ indica desde donde comienza el camino.
\subsubsection*{Camino en un Grafo Dirigido}
Importa quién se relaciona con quien. Ej.: $E = \{(A, B)_{\rightarrow}, (B, C)_{\rightarrow}\}$
\subsubsection*{Camino en un Grafo no Dirigido}
No importa quién se relaciona con quien. Ej.: $E = \{(A, B), (B, C)\}$. El camino podría ser en cualquier sentido. 
\subsection*{Ciclos}
Los ciclos son caminos tal que $v_{0} = v_{r}$. Solo existe en grafos no simples. 
\begin{itemize}
    \item $E = \{(A,B)_{\rightarrow}, (B, C)_{\rightarrow}, (C, A)_{\rightarrow} \}$ es un ciclo.
\end{itemize}
Una forma sencilla de pensarlo es lo siguiente: todo vértice recibe a un vértice, y todo vértice emite a un vértice. En algún momento se cierra la relación y los junta a todos.
\subsection*{Grafo Conexo}
Un Grafo Conexo es un Grafo que tiene la particularidad donde todos sus vértices están conectados entre sí. 
\begin{itemize}
    \item $E = \{(A,B)_{\rightarrow}, (B, C)_{\rightarrow}, \}$ es un grafo conexo.
    \item $E = \{(A, B)_{\rightarrow}, (C, D)_{\rightarrow}\}$ no es un grafo conexo.
\end{itemize}
\subsubsection*{Grafo fuertemente Conexo}
Un Grafo fuertemente conexo es un grafo dirigido y conexo. Todos los nodos deben tener una relación simétrica entre ellos.
\begin{itemize}
    \item $E = \{(A,B)_{\rightarrow}, (B, C)_{\rightarrow}\}$ es un grafo conexo.
    \item $E = \{(A, B)_{\rightarrow}, (B, C)_{\rightarrow}, (C, B)_{\rightarrow}, (B, A)_{\rightarrow},\}$ es un grafo fuertemente conexo.
\end{itemize}
\subsection*{Grafo T}
Es un árbol que cumple con las siguientes propiedades
\begin{itemize}
    \item Es conexo.
    \item No tiene ciclos. 
\end{itemize}
\subsection*{Grafo Bosque}
Es un árbol que cumple con la propiedad de que \textbf{no tiene ciclos}.
\end{document} 

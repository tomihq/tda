\documentclass[10pt,a4paper]{article}
\usepackage{blindtext}
\usepackage{subcaption}
\usepackage{graphicx}
\usepackage{tikz}
\usepackage{amssymb}
\usepackage{caption}
\usepackage{amsmath}
\usepackage{circuitikz}
\usepackage{hyperref}
\usepackage{amssymb}
\usepackage{amsmath}
\usepackage{listings}

\lstset{
    inputencoding=utf8,
    extendedchars=true,
    literate={á}{{\'a}}1 {é}{{\'e}}1 {í}{{\'i}}1 {ó}{{\'o}}1 {ú}{{\'u}}1 {ñ}{{\~n}}1 {Á}{{\'A}}1 {É}{{\'E}}1 {Í}{{\'I}}1 {Ó}{{\'O}}1 {Ú}{{\'U}}1 {Ñ}{{\~N}}1
}
\input{AEDmacros}
\newcommand{\notimplies}{\;\not\!\!\!\implies}
\title{Técnicas y Diseños de Algoritmos}
\author{Tomás Agustín Hernández}
\date{}

\begin{document}
\maketitle

\begin{figure}[b]
    \centering
    \begin{tikzpicture}[remember picture,overlay]
        \node[anchor=south east, inner sep=0pt, xshift=-1cm, yshift=2cm] at (current page.south east) {
            \begin{minipage}[b]{0.5\textwidth}
                \includegraphics[width=\linewidth]{logo_uba.jpg}
                \label{fig:bottom}
            \end{minipage}
        };
    \end{tikzpicture}
\end{figure}
\newpage
\section*{Grafos}
Un grafo es una estructura de datos compuesta por \textbf{V un conjunto de vértices (nodos)} y \textbf{E un conjunto de aristas que conectan pares de vértices}. \\
Definimos un grafo como: $G = (V, E)$ donde $E \subseteq V X V$.
\[\begin{minipage}[b]{0.7\textwidth}
    \includegraphics[width=\linewidth]{assets/graph.png}
\end{minipage}\]
Nótese que tiene sentido que el conjunto de aristas E esté formado por un par $V X V$. De lo contrario no existiría una arista. \\
\textbf{Good to Know} 
\begin{itemize}
    \item Si dos vértices están conectados por una arista, entonces estos se llaman \textbf{adyacentes}.
    \item La relación entre dos vértices define a $e = (u,v) \in E$. En este caso, decimos que \textbf{e} es incidente a \textbf{u} y \textbf{v}.
    \item La cantidad de vértices la notamos con la letra \textbf{n} y la cantidad de aristas con la letra \textbf{m}.
\end{itemize}
\subsection*{¿Por qué le decimos vértices a los nodos?}
Porque los grafos tienen un fundamento mucho más matemático.
\subsection*{Self Loop}
En un grafo, hablamos de self loop cuando un vértice se relaciona consigo mismo.
\subsection*{Cantidad de vertices y cantidad de aristas}
Definimos la cantidad de vértices de un grafo como $\longitud{V} = n$ y la cantidad de aristas como $\longitud{E} = m$. \\
\textbf{Good to Know}: En un grafo T (árbol) \textbf{siempre} sucede que $\longitud{V} > \longitud{E}$ en una unidad. Lo veremos más adelante, pero está bueno notarlo.
\subsection*{Vecindad de un vértice}
Llamamos vecindad de un vértice v a todos los vértices adyacentes de v. \\
Formalmente, lo definimos como: $N(v) = \{u \in V : (v, u) \in E\}$ 
\[\begin{minipage}[b]{0.5\textwidth}
    \includegraphics[width=\linewidth]{assets/vecindad.png}
\end{minipage}\]
Ej.: $N(2) = \{1, 3, 7, 6\}$ \\
\textbf{Good to Know}: Se utiliza la letra N de Neighbor (vecino) en inglés.
\subsection*{Grafo Simple}
Un Grafo Simple es un grafo que no tiene más de una arista entre dos mismos nodos. Nótese que aquí no importa la relación entre ellos, sino que, no debería estar repetido el mismo par (sin importar el orden).
\begin{itemize}
    \item $E = \{(A,B), (A,B)\}$ NO es un Grafo Simple.
    \item $E = \{(A,B), (B,A)\}$ o $E = \{(B,A), (A,B)\}$ NO es un Grafo Simple.
    \item $E = \{(A,B)\}$ SÍ es un Grafo Simple.
\end{itemize}
\subsection*{Grafo no Simple}
Un Grafo no Simple es un grafo que no tiene ninguna restricción con respecto a la relación entre dos vértices (nodos). 
\begin{itemize}
    \item $E = \{(A,B), (A,B)\}$ es un Grafo no Simple.
    \item $E = \{(A,B), (B,A)\}$ o $E = \{(B,A), (A,B)\}$ es un Grafo no Simple.
    \item $E = \{(A,B)\}$ SÍ es un Grafo no Simple.
\end{itemize}
\textbf{Good to Know}: Los grafos simples, son subconjuntos de grafos no simples. Por ende, si tenemos un grafo no simple que cumple las propiedades de un grafo simple, consideramos que es un \textbf{grafo simple} al ser más restrictivo. \\
En los ejemplos anteriores, entonces, $E = \{(A,B)\}$ sería mejor considerado como Grafo Simple.
\subsection*{Multigrafo}
Un Multigrafo es un tipo de \textbf{grafo no simple} el cual puede haber más de una arista entre el mismo par de vértices. 
\[\begin{minipage}[b]{0.5\textwidth}
    \includegraphics[width=\linewidth]{assets/multigrafo.png}
\end{minipage}\]
En el gráfico anterior, se definió el siguiente multigrafo: $G = \{\textbf{(1, 2), (1, 2), (1, 2)}, (2, 3), (3, 4), (3, 2), (4, 3) ...\}$
\subsection*{Pseudografo}
Un Pseudografo es un tipo de \textbf{grafo no simple} el cual tiene la particularidad de que puede haber más de una arista entre el mismo par de vértices, y, además, los vértices pueden tener self-loops. \\
\textbf{Good to Know}: Puede haber más de un self-loop de un vértice.
\subsection*{Grado de un vértice (nodo)}
El grado de un vértice es la cantidad de conexiones que tiene un nodo. \\
Definimos formalmente al grado de un vértice como $deg(v) = \#\{e \in E : v \in e\}$ \\
\textbf{Good to Know}: deg refiere a degree. \\
\textbf{Good to Know 2}: \textbf{e} es una arista particular $e = (v, w)$.\\
\textbf{Good to Know 3}: En un grafo G, al grado mínimo lo notamos como $\delta(G)$ mientras que al grado máximo lo notamos como $\triangle(G)$
\[\begin{minipage}[b]{0.5\textwidth}
    \includegraphics[width=\linewidth]{assets/max_min_grado.png}
\end{minipage}\]
En este caso particular, tenemos que $\delta(G) = 2$ y $\triangle(G) = 4$, y, a modo de ejemplo, $deg(2) = 4$ \\
\textbf{Good to Know 4}: Por cada par de vértices, tenemos una suma de grado 2. Ej.: $\{(1,2)\}$ nos da $deg(1) + deg(2) = 2$
\subsubsection*{Grado de un vértice en un Multigrafo}
Exactamente igual que en un Grafo, pero acá hay que aclarar que como podemos tener una relación entre dos vértices más de una vez tenemos que contar cada arista. Por ejemplo, si tenemos $\{(1,2), (1,2)\}$ entonces $deg(1) = 2$.
\subsubsection*{Grado de un vértice en un Pseudografo}
Exactamente igual que en un Grafo, pero acá el self-loop cuenta 2 veces. Por ejemplo, si tenemos $\{(1,2), (1,2), (1, 1), (1, 1)\}$ entonces $deg(1) = 6$. Si en algún momento te maréas, el self-loop vale dos porque tenés el mismo número en el par.
\subsection*{Suma de los grados de los vértices}
Dado un grafo G = (V, E) la \textbf{la suma de los grados} de sus vértices es igual a 2 veces el número de aristas. Es decir, 
\[\sum_{v \in V} deg(v) = 2 \times \#\{e \in E\} = 2m\] 
Para ver acerca de cómo demostrarlo, véase \hyperref[subsubsec:suma_grados_vertices]{\textbf{anexo}} \\
\textbf{Corolario}: En un grafo, siempre hay una cantidad par de vértices que tienen grado impar. Ej.: $\{(1, 2), (2, 3)\}, deg(1) = 1, deg(2) = 2, deg(3) = 1$
\subsection*{Grafo Dirigido (Digrafo)}
Un Grafo Dirigido es un tipo de grafo (que puede ser simple o no simple) pero importa quién se relaciona con quién, es decir, el orden en que guardamos la relación. En dibujos, lo notamos con una flecha. 
\begin{itemize}
    \item $E = \{(A,B)_{\rightarrow}, (B, A)_{\leftarrow} \}$ es un grafo dirigido, esto quiere decir que $(A,B) \neq (B, A)$
\end{itemize}
\textbf{Good to Know}: A nivel estructura de datos, parece que E es igual en un Grafo Dirigido que un Grafo No Dirigido. No obstante, optamos, a nivel de notación incluir una fecha para indicar la relación entre ambos. A nivel código, deberías manejar alguna información extra para entender si es dirigido o no.
\subsubsection*{Grado de un vértice en un Grafo Dirigido}
En un Grafo Dirigido, el grado de un vértice se calcula como $deg(v) = indeg(v) + outdeg(v)$ \\
¿A qué nos referimos con indeg(v) y outdeg(v)? Como es un grafo dirigido, la relación entre los vértices importa. Por lo tanto, aquellas aristas entrantes al nodo v origen las agrupamos en indeg(v) mientras que aquellas aristas salientes desde el nodo v las agrupamos en outdeg(v).
\begin{itemize}
    \item $E = \{(A,B)_{\rightarrow}, (B, A)_{\leftarrow}, (A, C)_{\rightarrow} \}$ es un grafo dirigido. Siendo indeg(A) = 1, outdeg(A) = 2 $\implies deg(A) = 3$ 
\end{itemize}
\subsubsection*{Cantidad de Aristas de un Grafo Dirigido Simple}
Como acá importa la relación los vértices pues contamos todas las aristas pues $(A,B) \neq (B,A)$ \[0 \leq |E| \leq n(n-1)\]
\subsubsection*{Cantidad de Aristas de un Grafo Dirigido NO Simple}
No hay límite, pueden ser infinitas. No hay fórmula concreta.
\subsection*{Grafo No Dirigido}
Un Grafo No Dirigido es un grafo el cual no existe la dirección en una relación entre dos vértices.
\begin{itemize}
    \item $E = \{(A,B), (B, A) \}$ es un grafo no dirigido, esto quiere decir que $(A,B) = (B, A)$
\end{itemize}
\textbf{Good to Know}: Los Grafos Dirigidos o Grafos no Dirigidos pueden ser Grafos Simples o no Simples.
\subsubsection*{Cantidad de Aristas de un Grafo NO Dirigido Simple}
Como no importa la relación entre los vértices pues $(A,B) = (B,A)$, contamos solo una vez esa arista \[0 \leq |E| \leq \frac{n(n-1)}{2} = \binom{n}{2}\]
\subsubsection*{Cantidad de Aristas de un Grafo NO Dirigido NO Simple}
No hay límite, pueden ser infinitas. No hay fórmula concreta.
\subsubsection*{Grado de un vértice en un Grafo No Dirigido}
Es la misma cuenta que hacemos en un grafo dirigido, pero sin distinguir en grupos a las aristas \textbf{(porque no tienen dirección)}.
\subsection*{Grafo Completo}
Llamamos Grafo Completo a un Grafo que tiene todos sus vértices están conectados entre sí. \\
Formalmente, lo notamos como $K_{n}$ donde \textbf{n} es la cantidad de vértices que tiene el grafo.
\[\begin{minipage}[b]{0.4\textwidth} 
    \includegraphics[width=\linewidth]{assets/grafo_completo.png}
\end{minipage}\]
\subsection*{Grafo Complemento}
Un Grafo Complemento ($G^{C}$) es el Grafo que tiene todas las \textbf{aristas} que no estaban en G. \\
Ej.: $G = \{(1, 2), (3, 2)\}$ entonces $ G^{c} = \{(1, 3), (2, 3)\}$
\subsubsection*{Cantidad de aristas de un Grafo Complemento}
\[m_{\overline{G}} = \frac{n(n-1)}{2} - m\]
\textbf{Creo que es necesario que sea un Grafo simple no dirigido}
\subsection*{Recorrido}
Un recorrido es una \textbf{sucesión de vértices} y \textbf{aristas} de un grafo, tal que $e_{i}$ sea incidente a $v_{i-1}$ y $v_{i}$ para todo $i = 1...k : P = v_{0}e_{1}v_{1}e_{2}...e_{k}v_{k}$ (vértice, par de vértices, vértice, par de vértices, vértice...). \\
Ej.: $P = 1 \rightarrow (1,2) \rightarrow 2 \rightarrow (2,3) \rightarrow 3 \rightarrow (3,4) \rightarrow 4 \rightarrow (4,6) \rightarrow 6 \rightarrow (2,6) \rightarrow 2 ...$ \\
Es decir, el nodo anterior (si hay) debe estar en la siguiente arista en alguna de las posiciones.
\subsection*{Camino}
Un camino es un recorrido \textbf{que no pasa dos veces} por el mismo vértice. \\
Ej.: $P = 1 \rightarrow 2 \rightarrow 3 \rightarrow 4 \rightarrow 5 \rightarrow 6 \rightarrow 7 \rightarrow 8 \rightarrow 9$
\[\begin{minipage}[b]{0.4\textwidth}
    \includegraphics[width=\linewidth]{assets/camino.png}
\end{minipage}\]
\subsection*{Sección (subconjunto/parte de un recorrido)}
Una sección es un tramo del recorrido P, se nota $P_{v_{i}}{v_{j}}$. Nótese que P es el recorrido, y los subíndices es aquello que acota al mismo. \\
Ej.: $P_{2,6} = 2 \rightarrow 3 \rightarrow 4 \rightarrow 6$. En este ejemplo, $v_{i} = 2$ y $v_{j} = 6$.
\[\begin{minipage}[b]{0.6\textwidth}
    \includegraphics[width=\linewidth]{assets/sección.png}
\end{minipage}\]
\textbf{Good to Know}
\begin{itemize}
    \item Una sección es un subconjunto de un recorrido P.
\end{itemize}
\subsection*{Circuito}
Un circuito es un recorrido que empieza y termina en el mismo vértice, se nota $P_{v_{i}}{v_{j}}$ \\
Ej.: $P_{1,1} = 1 \rightarrow 2 \rightarrow 3 \rightarrow 4 \rightarrow 6 \rightarrow 2 \rightarrow 1$
\[\begin{minipage}[b]{0.6\textwidth}
    \includegraphics[width=\linewidth]{assets/circuito.png}
\end{minipage}\]
\textbf{Good to Know}:
\begin{itemize}
    \item Un circuito es un tipo de recorrido. 
    \item Un circuito es opuesto a un camino. Porque un camino no tiene repetidos.
\end{itemize}
\subsection*{Ciclo (o circuito simple)}
Un ciclo o circuito simple es un circuito (de tres o más vértices) que no repite vértices, se nota $P_{v_{i}}{v_{j}}$.
Ej.: $P_{1,1} = 1 \rightarrow 2 \rightarrow 3 \rightarrow 4 \rightarrow 6 \rightarrow 7 \rightarrow 8 \rightarrow 9 \rightarrow 1$
\[\begin{minipage}[b]{0.6\textwidth}
    \includegraphics[width=\linewidth]{assets/ciclo.png}
\end{minipage}\]
\subsubsection*{Longitud}
La longitud de un recorrido P denotada l(P), es la \textbf{cantidad de aristas} que tiene. \\
$Ej.: P = 1 \rightarrow 2 \rightarrow 3 \rightarrow 4 \rightarrow 6 \rightarrow 2 \rightarrow 7$ donde $l(P) = 6$
\textbf{Good to Know}: Esta definición es muy importante, porque tranquilamente podría haber dos recorridos diferentes entre cada par de vértices.
\subsubsection*{Distancia entre dos vértices}
La distancia entre dos vértices u y v se define como la \textbf{longitud del recorrido más corto} entre u y v. Se denota $d(u,v)$ 
\[\begin{minipage}[b]{0.7\textwidth}
    \includegraphics[width=\linewidth]{assets/distancia.png}
\end{minipage}\]
\textbf{Good to Know} 
\begin{itemize}
    \item Precondiciones
    \begin{itemize}
        \item Tener dos vectores: u y v que pertenezcan a un grafo G.
        \item Tener uno o más recorridos, que incluyan los nodos de G.
    \end{itemize}
    \item Postcondiciones
    \begin{itemize}
        \item d(u,v) es el recorrido más corto desde u a v.
        \item Si $u=v \iff d(u,v) = 0$
        \item Si no existe un recorrido entre u y v entonces $d(u,v) = \infty$
    \end{itemize}
\end{itemize}
Véase \hyperref[subsubsec:distancia_demostracion]{\textbf{anexo}} para ver como demostrarlo por absurdo.
\subsubsection*{Camino en un Grafo Dirigido}
Importa quién se relaciona con quien. Ej.: $E = \{(A, B)_{\rightarrow}, (B, C)_{\rightarrow}\}$
\subsubsection*{Camino en un Grafo no Dirigido}
No importa quién se relaciona con quien. Ej.: $E = \{(A, B), (B, C)\}$. El camino podría ser en cualquier sentido. 
\subsection*{Subgrafo}
Sea $G = (V_{G}, E_{G})$ y $H = (V_{H}, E_{H})$ 
\subsubsection*{Subgrafo (base)}
H es un subgrafo de G si todos los vértices y aristas de H están en G. \\
Esto quiere decir que con que no haya aristas (conexiones entre vértices) o vértices que no estén en G, H es un subgrafo de G.
\[\begin{minipage}[b]{0.8\textwidth}
    \includegraphics[width=\linewidth]{assets/subgrafo_base.png}
\end{minipage}\]
\textbf{A partir de acá, los demás subgrafos cumplen mínimamente estas condiciones}
\subsubsection*{Subgrafo propio}
H es un subgrafo propio de G si todos los vértices y aristas de H están en G siempre y cuando \textbf{no sea la totalidad}. Es decir, no debe suceder que sea exactamente $H = G$.
\[\begin{minipage}[b]{0.8\textwidth}
    \includegraphics[width=\linewidth]{assets/subgrafo_propio.png}
\end{minipage}\]
\subsubsection*{Subgrafo Generador}
H es un subgrafo generador de G si H \textbf{tiene exactamente los mismos vértices} que G. 
\[\begin{minipage}[b]{0.8\textwidth}
    \includegraphics[width=\linewidth]{assets/subgrafo_generador.png}
\end{minipage}\]
\textbf{Un subgrafo generador es más fuerte que un subgrafo propio porque nos hacen usar todos los nodos pero conectarlos como queramos (siempre y cuando esas conexiones existan en el grafo original). Un subgrafo propio solo tiene una condición, no ser igual al grafo original.}
\subsubsection*{Subgrafo Inducido}
H es un subgrafo inducido de G si todos los nodos que están en H tienen todas las aristas de G que contienen exactamente a los nodos de H. 
Ej.: $G = ((1, 2, 3), \{(1,3), (1,2)\})$ si $H = (1,2)$ entonces está obligado a tener las aristas que contengan exactamente a los elementos 1 y 2. Por lo tanto serían: $\{1, 2\}$ 
\subsection*{Grafo Conexo}
Un Grafo Conexo es un grafo el cual existe un \textbf{camino} que conecta todos los vértices entre sí. 
\begin{itemize}
    \item $E = \{(A,B)_{\rightarrow}, (B, C)_{\rightarrow}, \}$ es un grafo conexo.
    \item $E = \{(A, B)_{\rightarrow}, (C, D)_{\rightarrow}\}$ no es un grafo conexo.
\end{itemize}
\subsubsection*{Componente Conexa}
Una componente conexa es un subgrafo.
\[\begin{minipage}[b]{0.9\textwidth}
    \includegraphics[width=\linewidth]{assets/componente_conexa.png}
\end{minipage}\]
Normalmente se consiguen cuando desconectás alguna arista de un grafo conexo. Entonces, te quedan subgrafos. La cantidad de subgrafitos son los componentes conexos.
\subsubsection*{Grafo fuertemente Conexo}
Un Grafo Fuertemente Conexo es un grafo dirigido y conexo. Todos los nodos deben tener una relación simétrica entre ellos.
\begin{itemize}
    \item $E = \{(A,B)_{\rightarrow}, (B, C)_{\rightarrow}\}$ es un grafo conexo.
    \item $E = \{(A, B)_{\rightarrow}, (B, C)_{\rightarrow}, (C, B)_{\rightarrow}, (B, A)_{\rightarrow},\}$ es un grafo fuertemente conexo.
\end{itemize}
\subsection*{Árboles}
\subsubsection*{Base}
Es un grafo conectado, sin ciclos (circuitos simples).
\[\begin{minipage}[b]{0.6\textwidth}
    \includegraphics[width=\linewidth]{assets/arbol_grafo.png}
\end{minipage}\]
\subsubsection*{Quitar Arista}
Si quitamos una arista cualquiera de un árbol, tenemos dos subgrafos conexos. Esto porque inicialmente un árbol no tiene ninguna componente suelta en el aire. 
\[\begin{minipage}[b]{0.6\textwidth}
    \includegraphics[width=\linewidth]{assets/arbol_quitar_arista.png}
\end{minipage}\]
\subsubsection*{Agregar Arista}
Si agrego una arista cualquiera se forma un ciclo (circuito simple).
\[\begin{minipage}[b]{0.8\textwidth}
    \includegraphics[width=\linewidth]{assets/arbol_agregar_arista.png}
\end{minipage}\]
\subsection*{Grafo T}
Es un árbol que cumple con las siguientes propiedades
\begin{itemize}
    \item Es conexo.
    \item No tiene ciclos. 
\end{itemize}
\subsection*{Grafo Pesado}
Definimos un Grafo Pesado G como $G = (V, E, W)$ donde W es una función que recibe dos vértices y devuelve el peso. \\
Ahora, en cada arista almacenamos ambos vértices y el resultado del peso. 
\[E_{i} = (v_{i}, v_{j}, w(v_{i},v_{j}))\]
\subsection*{Grafo Acíclico Dirigido (DAG)}
Un Grafo Acíclico Dirigido es un grafo dirigido que no tiene ciclos. Es decir, no existe ningún vértice que es capaz de empezar y terminar un recorrido. \\
Ej.: Una Single Linked List es un caso particular de un DAG.
\subsection*{Grafos Bipartito}
Dado un grafo $G = (V, E)$, G es bipartito si
\begin{itemize}
    \item Existe $V_{1}$ y $V_{2}$ tal que $V_{1} \cup V_{2} = V$
    \item $V_{1} \cap V_{2} = \emptyset$
    \item $e = (u,v) \in E, u \in V_{1}, v \in V_{2}$. Esto significa que el vértice de $V_{1}$ se relaciona con el de $V_{2}$ o en el otro sentido. 
    \item La única restricción es que no podés conectar un vértice de $V_{1}$ con otro de $V_{1}$ ni tampoco uno de $V_{2}$ con otro de $V_{2}$
\end{itemize}
\subsection*{Grafo Bipartito Completo}
Lo mismo que un Grafo Bipartito pero acá todos los vértices de $V_{1}$ están conectados con los de $V_{2}$ y viceversa. \\
Básicamente, el producto cartesiano.
\subsubsection*{Cosas a releer / preguntar}
\begin{itemize}
    \item Un grafo es bipartito, $V = (V_{1}, V_{2}) \iff$ no tiene ciclos de longitud impar.
    \item Un grafo es bipartito $\iff$ todas sus c.c. lo son 
    \item Un grafo no tiene ciclos impares $\iff$ cada una de sus c.c. no tienen ciclos impares.
\end{itemize}
\subsection*{Isomorfismo de Grafos}
Dados  $G = (V, E)$ y $G' = (V', E')$. $G$ y $G'$ son isomorfos si existe una función biyectiva $f: V \rightarrow V$ tal que $\forall u,v \in V$ 
\[(u,v) \in E \iff (f(u), f(v)) \in E'\]
Es decir, \textbf{si parto desde E y agarro el par que está en una arista, puedo poner cada componente en una función f y me lo transforma en una arista que está en E'}. \\
Lo mismo con el caso opuesto. \\
\textbf{Good to Know}: En nuestro caso, si G y G' son isomorfos los vamos a notar como $G = G'$ porque son lo mismo pero cambian la forma en que se nombran los nodos y el orden de las aristas.
\subsubsection*{Propiedades}
Si dos grafos $G = (V,E)$ y $G' = (V', E')$ son isomorfos entonces
\begin{itemize}
    \item Tienen el mismo número de vértices (sí, porque en la biyectividad se cumple la inyectividad y la sobreyectividad).
    \item Tienen el mismo número de aristas (sí, porque cada una debería traducirse a una única específica, de lo contrario no sería inyectiva).
    \item Para todo $k$, $1 \le k \le n-1$, tienen el mismo número de vértices de grado k (la misma distribución de grado).
    \item Tienen la misma cantidad de componentes conexas.
    \item Para todo k, $1 \le k \le n-1$, tienen el mismo número de caminos simples de longitud k.
\end{itemize}
\[\begin{minipage}[b]{0.8\textwidth}
    \includegraphics[width=\linewidth]{assets/isomorfos.png}
\end{minipage}\]
\subsection*{Representación de grafos}
\textbf{Good to Know}: Si busca uso de grafos en la práctica, y cómo representarlos véase \hyperref[subsec:representacion_grafos_practica]{\textbf{anexo}}.
\textbf{Good to Know}: En las diapositivas se usa la $d$ para representar tanto como el grado de un vértice o la distancia entre dos vértices. Acá, optamos por usar $deg$ para el grado y $d$ para distancia entre dos vértices.
\subsubsection*{Lista de Adyacencia}
Es la más común y la más eficiente. \\
Es una lista de listas, donde cada lista corresponde al vértice y las aristas con las que se conecta se representa con una lista de diccionarios. 
\[\begin{minipage}[b]{0.8\textwidth}
    \includegraphics[width=\linewidth]{assets/adj_list.png}
\end{minipage}\]
\subsubsection*{Matriz de Adyacencia}
Cada número está representado por el índice en donde está. Vale 1 \textbf{(o el valor de peso de la arista)} si existe la arista y 0 si no. \\
La cantidad de filas y columnas es igual a la cantidad de nodos que hay, es por eso que espacialmente cuesta $O(n^{2})$
\[\begin{minipage}[b]{0.8\textwidth}
    \includegraphics[width=\linewidth]{assets/matriz_ady.png}
\end{minipage}\]
Ej.: El nodo 3 (fila 3), está relacionado con todos los nodos que están en la columna 3. En el dibujo, el nodo 3 está relacionado con el 2 y con el 4. Por eso, $a_{3, 2}$ = 1 y $a_{3, 4}$. 
\[\begin{minipage}[b]{0.8\textwidth}
    \includegraphics[width=\linewidth]{assets/adj_matrix.png}
\end{minipage}\]
\textbf{Good to Know}
\begin{itemize}
    \item En el dibujo vemos que empieza todo desde el índice 1 pero en realidad empezamos desde el 0. 
    \item \textbf{Lo que todavía no entiendo es, qué pasaría si los nodos guardan información compleja, como elijo cual va en el índice 0, cual en el 1, etc.} 
    \item La matriz es simétrica si el grafo es no dirigido. Es decir, ver desde fila/columna o columna/fila es lo mismo. 
\end{itemize}
Esta forma de trabajar con grafos es \textbf{súper ineficiente}. 
\subsubsection*{Algunas propiedades útiles (solo grafos no dirigidos)}
\begin{itemize}
    \item 1. La suma de los elementos de la fila i (o columna j) es igual a $deg(u_{i})$, es decir: \textbf{$\sum_{i=0}^{j} a[i][0] = \sum_{j=0}^{i} a[j][0]$}
    \item 2. Los elementos de la diagonal principal de $A^{2}$ indican los grados de los vértices $a^{2}_{ii} = deg(u_{i})$. \textbf{Nota}: usa ${ii}$ porque hay misma cantidad de filas que de columnas, de lo contrario, se podría decir ${ij}$ con $i=j$. Esto es más que nada para hablar de los índices de la diagonal principal de una matriz.
\end{itemize}
1. Nótese que en la columna $a_{i, ultColumna + 1}$ o en la fila $a_{ultFila + 1, j}$ se encuentran las sumas de esa fila o columna.
\[\begin{minipage}[b]{0.9\textwidth}
    \includegraphics[width=\linewidth]{assets/suma_matriz_ady.png}
\end{minipage}\]
2. Nótese que ahora lo que estaban en la fila / columna extra, ahora se pasaron al $a_{ii}$ (o $a_{ij}$ con $i=j$)  correspondiente.
\[\begin{minipage}[b]{0.6\textwidth}
    \includegraphics[width=\linewidth]{assets/diagonal_principal_ady.png}
\end{minipage}\]
\subsection*{Matriz de Adyacencia con Grafos Dirigidos}
\begin{itemize}
    \item La matriz de adyacencia de un grafo dirigido NO es simétrica como sí pasaba con un grafo común.
    \item G = (V, E) donde E es un conjunto de pares ordenados. Esto significa que el \textbf{orden de los elementos del par} importan.
    \item $e=(u,v)$ también se llama \textbf{arco}. A $u$ se le llama \textbf{cola} y a $v$ se le llama \textbf{cabeza}. (medio raro, pero bueno).
    \item Los digrafos tienen grados de entrada $(deg_{in})$ y grados de salida $(deg_{out})$
    \item El grafo subyacente $(G^{s})$ es el que resulta de remover las direcciones.
    \item La suma de los elementos de la fila i es igual a $d_{OUT(u_{i})}$.
    \item La suma de los elementos de la columna j es igual a $d_{IN(u_{j})}$
\end{itemize} 
\section*{BFS}
Breadth-First Search o mejor conocido como BFS es un algoritmo de búsqueda para grafos. \\
Casos de uso: 
\begin{itemize}
    \item Se usa en grafos no ponderados (no tiene peso en aristas).
    \item Se usa para encontrar el camino más corto desde un nodo a otro. 
    \item Se usa para asegurarnos que siempre pasamos por todos los nodos del mismo nivel antes de pasar a otro. 
    \item Explora el grafo o árbol nivel a nivel, primero horizontalmente y luego verticalmente. Es decir, primero agarra todos los vecinos, después del primer vecino baja al siguiente nivel, después del segundo vecino baja al siguiente nivel y así sucesivamente. 
    \item A nivel estructura de datos se implementa con una \textbf{queue} (FIFO).
    \item Se almacena en una lista los nodos visitados para no repetirlos. Es decir, que antes de pasar por uno hay que chequear si ya está.
\end{itemize}
\subsection*{Complejidad temporal de BFS}
La complejidad temporal de BFS es $O(\longitud{V} + \longitud{E})$. 
\section*{DFS}
Depth-First Search o mejor conocido como DFS es un algoritmo de búsqueda para grafos. \\
Casos de uso: 
\begin{itemize}
    \item Explora todos los nodos a profundidad, primero verticalmente y luego horizontalmente.
    \item A nivel estructura de datos se implementa con un \textbf{stack} (LIFO).
    \item Se usa en grafos no ponderados (no tiene peso en aristas).
\end{itemize}
Un ejemplo claro de DFS sería buscar la altura de un árbol porque tendrías que buscar la rama más larga.
\subsection*{Complejidad temporal de DFS}
La complejidad temporal de DFS es $O(\longitud{V} + \longitud{E})$.
\section*{Fuerza Bruta}
Consiste en analizar y listar todas las posibilidades y quedarme con las que me interesan. \\
En la fuerza bruta, existen problemas de factibilidad y optimalidad. 
\begin{itemize}
    \item En los problemas de factibilidad lo que queremos son soluciones que cumplan ciertas características. 
    \item En los problemas de optimizaciones combinatorias, teniendo todas las soluciones factibles me quedo con la mejor (según el criterio que considero qué es mejor). 
\end{itemize}
Los algoritmos por fuerza bruta son muy fáciles de implementar pero son muy ineficientes; son útiles cuando tenemos instancias súmamente pequeñas. \\

Ej.: Si tenemos que completar un tablero de ajedrez tendríamos que poner todas las posibilidades en todos los casilleros y si hay pocas soluciones válidas estaríamos tardando demasiado tiempo por nada.
\section*{Backtracking}
Es una modificación de la técnica de fuerza bruta que aprovecha propiedades del problema para evitar analizar todas las configuraciones. \\
Para que el algoritmo sea correcto debemos estar seguros de no dejar de examinar configuraciones que estamos buscando. \\
Las soluciones candidatas se representan con un \textbf{vector a = $(a_{1}, a_{2}, \dots, a_{n})$}
\begin{itemize}
    \item Soluciones parciales: Son las soluciones que se van armando en cada paso. 
    \begin{itemize}
        \item Cada nodo es una solución parcial.
        \item La raíz del árbol es el vector vacío (solución parcial vacía).
    \end{itemize}
    \item Soluciones candidatas: Depende el problema. A veces son todas las hojas (si necesitás gastar todos los elementos), a veces te basta con cumplir X cosa y no necesitás recorrer todo el árbol. 
    \item Soluciones válidas: Son aquellas soluciones que cumplen con todas las características que buscamos. A veces son las hojas del árbol, a veces no es necesario llegar hasta las hojas, porque capaz se cumple en uno, o dos pasos tempranos.
\end{itemize}
\textbf{¿Todas las soluciones candidatas son soluciones parciales? Porque si una posible solución válida llegás en el primer nivel del árbol, entonces esa solución era candidata, pero también era parcial.} \\
Ej.: Ejercicio de armar subconjuntos que sumen n.  
\subsection*{Podas}
En el backtracking, recorremos el árbol en profundidad. \\
Cuando podemos ver que una solución parcial no nos llevará a una solución válida, no es necesario seguir explorando esa rama del árbol de búsqueda (se poda el árbol) y se retrocede hasta encontrar un vértice con un hijo válida por donde seguir. 
\begin{itemize}
    \item Poda por factibilidad: Ninguna extensión de la solución parcial derivará en una solución valida del problema. 
    \begin{itemize}
        \item Digamos que quiero llegar a armar subconjuntos que sumen 9, si con el nodo actual la suma es 10, entonces no me sirve seguir hacia adelante. Podo de una. Esto se llama \textbf{propiedad dominó}, si una solución parcial no vale, la generada a partir de ella menos todavía. 
        \item \textbf{¿Qué pasaría, si llegase a haber algún caso donde capaz nuestra solución parcial ya no es válida, pero si seguíamos adelante capaz teníamos un factor que la volvía a hacer válida? Por ej.: En un conjunto que tenes números positivos y negativos tenés que armar subconjuntos que sumen k. Si te pasaste de k, no te sirve podar de una porque capaz encontrás algun negativo que te ayude a llegar a k de vuelta.}. Respuesta: No te sirve esa poda para ese problema. 
    \end{itemize}
    \item Poda por optimalidad (problemas de optimización): Ninguna extensión de la solución parcial derivará en una solución del problema óptima.
\end{itemize}
Un problema de backtracking que no tiene podas es un problema de fuerza bruta.
\subsection*{Correctitud en Backtracking}
Para demostrar la correctitud de un algoritmo de backtracking, debemos demostrar que se enumeran todas las configuraciones válidas. Si hacés una poda, me tenés que garantizar que aunque el rendimiento haya mejorado, me sigas trayendo las mismas soluciones válidas. \\
Ej.: Si lo hacemos por fuerza bruta, podemos conseguir todas las soluciones válidas. Luego, si lo hacemos por backtracking y añadimos podas, nos fijamos que tengamos los mismos resultados que si lo hacemos por fuerza bruta.
\subsection*{Complejidad en Backtracking}
Consideramos el costo de procesar cada nodo. Cada nodo aparece cada vez que hacemos un llamado recursivo.  
\subsection*{Tips Backtracking}
\begin{itemize}
    \item Si tengo conjuntos, solo me interesan tener una única ocurrencia. Es decir, $\{1, 3\} = \{3, 1\}$ por lo tanto, siempre priorizo que el elemento de la izquierda pueda tener a todos los mayores de la derecha pero no al revés. Esto gana mucho tiempo porque estamos eliminando soluciones repetidas.
    \[\begin{minipage}[b]{0.9\textwidth}
    \includegraphics[width=\linewidth]{assets/backtracking_half_search.png}
    \end{minipage}\]
\end{itemize}
\section*{Programación Dinámica (DP)}
Se basa en solucionar y memorizar subproblemas que se repetirán para ahorrar tiempo. \\
Esa \textbf{superposición de problemas} es que estamos llamando usando algo que \textbf{ya resolvimos anteriormente}, entonces, para evitar calcular todo nuevamente, usamos la información almacenada en memoria. \\
Es \textbf{bastante común} usar este tipo de estrategias para afrontar problemas con matrices, o casos recursivos para cosas altamente costosas de calcular. \\
\textbf{Importante}: ¿Cuándo hay veces que la entrada no es exactamente igual pero vale la memorización para ese subproblema? \\
Cuando la dirección de los argumentos no importa, se puede utilizar ese subproblema memorizado, es decir, en casos donde, por ejemplo $(a, b) = (b, a)$.
\subsection*{Formas de realizar programación dinámica}
Se puede hacer tanto de forma Top-Down (recursivamente) o Bottom-Up (iterativamente). \\
En Top-Down hablamos de \textbf{memorización} mientras que en Bottom-Up hablamos de \textbf{tabulación}. 
\subsubsection*{Bottom-Up vs Top-Down}
\begin{itemize}
    \item Top Down (memorización): Usamos recursión, guardamos los resultados en una caché (generalmente diccionario o array). Solo resolvemos los subproblemas que necesitamos.
    \item Bottom Up (tabulación): Usamos un enfoque iterativo, contruimos la solución desde los casos base hacia arriba mientras que llenamos una tabla paso a paso. 
\end{itemize}
\[\begin{minipage}[b]{0.8\textwidth}
    \includegraphics[width=\linewidth]{assets/topdown_bottomup.png}
\end{minipage}\]
\subsection*{Programación Dinámica Top-Down}
\subsubsection*{Fibonacci y su relación con la Programación Dinámica}
\[\begin{minipage}[b]{0.8\textwidth}
    \includegraphics[width=\linewidth]{assets/fib_dp.png}
\end{minipage}\]
¿Donde está la superposición de problemas acá? Bueno, veamos que para \textbf{fib 5} se encuentra repetida una vez. \\
Entonces, la idea, es que guardemos el resultado para el primer \textbf{fib} que salga, y luego reutilizarla. \\
Nótese que al ser \textbf{fib(n-1)} + \textbf{fib(n-2)} en algún momento, indudablemente el \textbf{n-1} va a llegar a ser el valor que fue \textbf{n-2}, y ahí justamente, entra la memorización. \\
A nivel código, no es más que guardar en un array, map o una estructura eficiente. \\ 
Sin ningún tipo de memorización, la complejidad que da en tiempo es $O(n^{2})$ mientras que en espacio da $O(n)$. \\
\textbf{En espacio da $O(n)$} porque a nivel stack, lo más profundo que baja en una rama es n veces. 
\subsubsection*{Con Programación Dinámica}
Sabiendo que tenemos superposición de problemas basado en n, optimizado quedaría así
\[\begin{minipage}[b]{0.9\textwidth}
    \includegraphics[width=\linewidth]{assets/fib_dp_finish.png}
\end{minipage}\]
¿Cómo quedó la complejidad temporal y espacial? Bueno, espacialmente seguimos teniendo $O(n)$ porque bajamos hasta el caso base, al menos una vez, pero temporalmente ahora también tenemos $O(n)$
\[\begin{minipage}[b]{0.9\textwidth}
    \includegraphics[width=\linewidth]{assets/fib_dp_finish_o.png}
\end{minipage}\]
\subsubsection*{Llegando a un punto particular en una grilla}
Digamos que queremos movernos a un punto particular fijo y solo podemos movernos a la derecha y para abajo. Es sabido que van a existir muchas formas de llegar a ese punto. \\
Ahora bien, a medida que nos vamos acercando, el problema se achica. Así que, seguramente, vaya a existir algún otro camino por el que lleguemos al punto (o no). 
\[\begin{minipage}[b]{0.9\textwidth}
    \includegraphics[width=\linewidth]{assets/gridTraveler.png}
\end{minipage}\]
Sin ningún tipo de memorización, la complejidad que da en tiempo es $O(2^{n+m})$ porque para cada caso, tengo dos opciones que dependen de dos variables mientras que en complejidad espacial da $O(n+m)$ que es lo que nos cuesta bajar hasta el caso base gastando todas las opciones.
\subsubsection*{Con Programación Dinámica}
¿Cómo hacemos para guardar en una estructura de memorización algo que tiene más de un argumento? Normalmente, lo ideal es guardar algo de la especie \textbf{m + ', ' + n}. \\
Es realmente importante NO guardar todos de forma continua y sí distinguirlos por algún símbolo \textbf{(,)} por si en algún momento, tenemos que considerar los elementos separados pero que valgan por simetría.
\[\begin{minipage}[b]{0.9\textwidth}
    \includegraphics[width=\linewidth]{assets/gridTraveler_2.png}
\end{minipage}\]
Por último, veamos como mejoró nuestra solución\[\begin{minipage}[b]{0.9\textwidth}
    \includegraphics[width=\linewidth]{assets/grid_traveler_op.png}
\end{minipage}\]
\subsubsection*{Receta de Programación Dinámica (Top-Down)}
\begin{itemize}
    \item Hacer que la solución sin programación dinámica funcione
    \begin{itemize}
        \item Visualizar el problema como un árbol.
        \item Implementar ese árbol usando recursión.
        \item Testear si la solución funciona. 
    \end{itemize}
    \item Optimizarlo
    \begin{itemize}
        \item Agregar un objeto de memo.
        \item El objeto memo debe pasarse a todos los llamados recursivos. Normalmente lo pasamos por referencia o usamos variables globales.
        \item Agregar casos base para los valores memorizados.
        \item Guardar los valores de los return dentro del objeto memo.
    \end{itemize}
\end{itemize}
Para resolver problemas de programación dinámica se recomienda ser meticuloso, hacer dibujos, armar buenas soluciones de fuerza bruta y recién luego pensar la estrategia de memorización. Memorizar es sencillo, qué memorizar es complicado. \\
No traten de implementar una solución óptima desde el principio. Make it work first.
\subsubsection*{canSum problem}
La idea es que dado un array, se diga si existe algún subconjunto de elementos que suman un valor n particular. \\
\textbf{Tip importante}: En problemas donde hay que llegar a determinado valor k, siempre, pero siempre conviene disminuir ese valor de target y no ir sumando.
\[\begin{minipage}[b]{1\textwidth}
    \includegraphics[width=\linewidth]{assets/canSum.png}
\end{minipage}\]
Algo que siempre hay que tener en cuenta es que, si llegamos a determinado valor, el árbol siempre va a tener el mismo comportamiento para ese valor. Nótese que acá, en este problema particular tampoco nos interesa qué numeros suman nuestro target, sino, simplemente si hay un camino o no.
\subsection*{Programación Dinámica Bottom-Up}

\section*{Anexo}
\subsection*{Demostraciones de Grafos}
Normalmente se hace inducción sobre las aristas, aunque en algunos casos, optamos sobre los vértices. \\
Lo más común es tener un caso base del tipo $n=1, m=0$ o $n=2, m=1$. \\
\textbf{Good to Know}: Usar a favor las propiedades.
\subsubsection*{Distancia entre dos vértices}
\label{subsubsec:distancia_demostracion}
Si P es el recorrido entre u y v tiene longitud d(u,v) $\implies$ P es un camino. \\
Recordemos algunas definiciones
\begin{itemize}
    \item Un recorrido es una sucesión de vértices y aristas, que no tiene ninguna limitación.
    \item d(u, v) es el recorrido más corto entre u y v. Por lo que, podría haber otros. 
    \item P es un camino: Un camino es un subconjunto de un recorrido donde no puede haber vértices repetidos.
\end{itemize}
Probando por el absurdo: 
\[\neg Q \implies R\]
En este caso
\begin{itemize}
    \item $\neg Q$ = P no es un camino
    \item $R$ = P es el recorrido entre u y v y tiene longitud d(u,v)
\end{itemize}
La idea es llegar a algo falso.  
\[V \implies F = F\]
Suponemos que P NO es un camino $\neg Q = V$, entonces hay, al menos un vértice en P que se repite.  \\
Recorrido P \[u \rightarrow z \rightarrow z \rightarrow v\]
¿Podemos encontrar algún camino que sea más corto que P? (esto haría falso el consecuente). \\
Defino un camino T \[u \rightarrow z \rightarrow v\] 
Luego, la distancia mínima entre ambos recorridos
\[min(dP(u,v), dT(u,v)) = min(3, 2) = 2 = dT(u,v)\]
Finalmente, llegamos a un absurdo, porque la distancia mínima la tenemos con el recorrido T y no con P. \\
Si hubiesemos llegado a que con $\neg Q$, P es el camino más corto, hicimos mal la demostración.
\subsection*{Representación de grafos en la práctica}
\label{subsec:representacion_grafos_practica}
Tenemos dos formas de utilizar grafos, o usamos a través del conjunto de aristas (V, E) o a través del conjunto de vecindarios (N, V). Exploremos las ventajas de cada una y cuando usamos una y cuando la otra. \\
Hay diversos ítems que pueden contribuir a esta decisión, y la forma de es viendo si el grafo tiene algo especial. 
\subsubsection*{¿Qué cataloga a un grafo como especial?}
¿Qué necesito para que un grafo sea considerado como especial? Si el grafo tiene al menos una de las siguientes propiedades.
\begin{itemize}
    \item Es un árbol.
    \begin{itemize}
        \item No tiene ciclos. 
        \item Todos los vértices están conectados.
        \item Tiene exactamente \textbf{n-1} aristas y \textbf{n} nodos.
        \item Se puede representar solo con un padre por nodo, sin guardar todas las aristas.
    \end{itemize}
    \item Es denso o completo
    \item Tiene pesos o el peso se calcula con una fórmula 
    \item Es ralo y dirigido.
    \item Tiene simetrías o propiedades algebraicas conocidas
    \begin{itemize}
        \item Grafo cíclico, red de tipo grilla.
        \item Se puede describir con una fórmula o regla en vez de listar cada arista.
    \end{itemize}
    \item Es estático y accedés mucho a los vecinos.
\end{itemize}
\subsubsection*{Conjunto de Aristas (V, E)}
Es muy común cuando no conocemos \textbf{nada especial} sobre la estructura del grafo que podamos aprovechar. Es compacta y directa.  
\subsubsection*{Conjunto de vecindarios (V, N)}
N es una función que te dice, para cada vértice v, con qué otros vértices está conectado. La forma de utilizarlo varía si es un grafo dirigido o no dirigido. 
\begin{itemize}
    \item N(v): Devuelve los vecinos de v.
    \item En digrafos:
    \begin{itemize}
        \item $N^{+}(v)$ (vecindario de salida): a dónde puede ir v.
        \item $N^{-}(v)$ (vecindario de entrada): de dónde puede venir v.
    \end{itemize}
\end{itemize}
Esta representación es útil cuando \textbf{queremos recorrer el grafo desde los vértices}.
\subsubsection*{Numeración de los vértices}
Numeramos los vértices desde 0 hasta n-1 (donde n es la cantidad de vértices). Esto hace que sea más fácil de trabajar con arrays y matrices.
\subsubsection*{¿Qué representación usar? ¿Cuál es más eficiente?}
Si el grafo no tiene una estructura especial, usamos el conjunto de aristas (V, E) porque 
\begin{itemize}
    \item No ocupa más espacio del necesario.
    \item No asume nada: es la forma neutra de representar el grafo. 
\end{itemize}
Ahora, si el grafo tiene una estructura especial, usamos el conjunto de vecindarios (V, N) porque 
\begin{itemize}
    \item Es más eficiente.
    \item Permite recorrer el grafo de una forma más sencilla.
\end{itemize}
Decimos que la representación más fiel del conjunto de vecindarios es a través de las listas de adyacencias. Veamos un ejemplo a continuación. 
\subsection*{Ejemplo 1}
Queremos ver si podemos llegar desde la ciudad A hasta la ciudad D. Un cliente nos trae la siguiente información donde la primer letra corresponde al saliente y la segunda el destino. \\
\begin{lstlisting}
{
  "rutas": [
    ["A", "B"],
    ["B", "C"],
    ["C", "D"],
    ["A", "E"],
    ["E", "F"]
  ],
  "origen": "A",
  "destino": "D"
}
\end{lstlisting}
Acá es donde tenemos que pensar cómo armamos nuestro grafo. \\ 
\subsubsection*{Forma 1 (Lista de Adyacencia)}
Sabemos que tenemos que ir desde una ciudad a otra, asì que estarìa bueno tener la lista de conexiones de una ciudad con todas las demás. Es decir, meter todas las ciudades con las que se conecta una en una lista. \\
En este caso, nos conviene una lista de adyacencia (conjunto vecindario) y ahora veremos por qué.\\
Nos quedaría entonces, algo así 
\begin{lstlisting}
    graph = {
        "A": ["B", "E"],
        "B: ["C"],
        "C": ["D"],
        "E": ["F"],
    }
\end{lstlisting}
El problema entonces sería algo así: 
\begin{itemize}
    \item Empiezo en A, tomo cada vecino de él y me fijo con quién se conectan.
    \begin{itemize}
        \item Llegué a B
        \begin{itemize}
            \item B se conecta con C, C se conecta con D, entonces A se conecta con D. 
            \item B se conecta con E, E se conecta con F y F no se conecta con nadie.
        \end{itemize}
        \item Respuesta final: Sí existe un camino, y ese es A -> B -> C -> D y es un circuito simple.
    \end{itemize}
\end{itemize}
¿Cómo quedaría de esta manera, la complejidad temporal y espacial en el armado del grafo? 
\begin{itemize}
    \item Temporal:O(E) porque el input inicial es una lista de aristas y recorremos toda esa lista.
    \item Espacial: O(N + E) porque por cada nodo, guardamos una lista de sus vecinos.
\end{itemize}
¿Y en la búsqueda? 
\begin{itemize}
    \item Espacial: O(N) que es lo que cuesta ir guardando los nodos visitados. 
    \item Temporal: O(N + E) porque por cada nodo, vemos todas sus aristas.
\end{itemize}
¿Cuál es la complejidad total del algoritmo? En tiempo es O(N + E) y en espacio O(N + E). \\
\textbf{¿Cuando usar listas de adyacencias?}
\begin{itemize}
    \item El grafo es disperso, es decir, tiene más aristas que nodos.
    \item Es importante optimizar el espacio.
    \item Buscás los vecinos frecuentemente.
\end{itemize}
\subsubsection*{Forma 2 (Conjunto de Aristas)}
Si representaríamos el grafo como un conjunto de aristas, sería algo así:
\begin{lstlisting}
    graph = [(A, B), (B, C), (C, D), (A, E), (E, F)]
\end{lstlisting}
Y esto, es mucho más costoso que lo anterior. ¿Por qué? Porque acá no tenemos las claves de los nodos, y eso implica que para buscar todos los vecinos de un nodo tenemos que recorrer en tiempo O(E) \\
¿Cómo quedaría de esta manera, la complejidad temporal y espacial en el armado del grafo? 
\begin{itemize}
    \item Temporal: O(E) porque el input inicial es una lista de aristas y recorremos toda esa lista.
    \item Espacial: O(E) porque por cada nodo, guardamos una lista de sus vecinos.
\end{itemize}
¿Y en la búsqueda? 
\begin{itemize}
    \item Buscar todos los vecinos de A - Tiempo: O(E), Espacio: O(k) si los almacenamos en un array pero hay que recordar que en el peor caso, A es vecino de todos. Entonces sería O(E).
    \item Buscar todos los vecinos de B y E - Tiempo: O(E), Espacio O(E). 
    \item Buscar todos los vecinos de C y F - Tiempo: O(E), Espacio O(E).
    \item Encontramos la respuesta, porque C está conectado con D. 
\end{itemize}
¿Cuál es la complejidad total del algoritmo? En tiempo y espacio es O(E).
\textbf{¿Cuando usar conjuntos de aristas?}
\begin{itemize}
    \item El grafo dinámico y se modifican frecuentemente las aristas.
    \item Si necesitás buscar o contar la cantidad total de aristas.
    \item No necesitamos almacenar información sobre los vecinos de los nodos y quiero manipular aristas de manera eficiente.
\end{itemize}
\subsubsection*{Forma 3 (Matriz)}
Es importante recordar que en la matriz, la dimensión es siempre como si todos los nodos estarían relacionados con todos. \\
En este caso tenemos 6 nodos, y en el peor caso se conectan todos con todos. Así que la matriz sería de 6x6. \\
\begin{lstlisting}
    matrix = [
    [0, 1, 0, 0, 1, 0],  //A
    [0, 0, 1, 0, 0, 0],  //B
    [0, 0, 0, 1, 0, 0],  //C
    [0, 0, 0, 0, 0, 0],  //D
    [0, 0, 0, 0, 0, 1],  //E 
    [0, 0, 0, 0, 0, 0],  //F 
]
\end{lstlisting}
¿Cómo quedaría de esta manera, la complejidad temporal y espacial en el armado del grafo? 
\begin{itemize}
    \item Temporal: O($N^{2}$).
    \item Espacial: O($N^{2}$).
\end{itemize}
¿Y en la búsqueda? 
\begin{itemize}
    \item Buscar todos los vecinos de A - Tiempo: O(N).
    \begin{itemize}
        \item Buscar todos los vecinos B - Tiempo: O(N). Buscar todos los vecinos de C - Tiempo: O(N). 
        \item Buscar todos los vecinos de E - Tiempo: O(N). Buscar todos los vecinos de F - Tiempo: O(N).
    \end{itemize}
\end{itemize}
Como esto se hace varias veces, sería algo de O(cantVecinosNodo * N) = O(N). \\
¿Cuál es la complejidad total del algoritmo? En tiempo y espacio es O($N^{2}$). \\
\textbf{¿Cuando usar matrices de adyacencia?}
\begin{itemize}
    \item El grafo es denso.
    \item Necesitás verificar si existe rápidamente una conexión entre dos nodos.
    \item El grafo es estático, es decir, no se agregan ni se quitan aristas frecuentemente.
\end{itemize}
\subsubsection*{Comparación de las tres estructuras que vimos}
Si bien la complejidad parece ser mejor usando el conjunto de aristas, esto no es del todo cierto. ¿Por qué? Porque si necesitamos rápidamente los vecinos de un nodo particular, con listas de adyacencia los encontramos en O(1) y no en O(E). Esa es la diferencia sustancial que en grafos gigantes se nota la diferencia. 
\begin{table}[h!]
    \centering
    \begin{tabular}{|l|c|c|c|}
    \hline
    \textbf{Operación}                        & \textbf{Lista de Adyacencia}          & \textbf{Matriz de Adyacencia}      & \textbf{Conjunto de Aristas}     \\ \hline
    \textbf{Espacio (Construcción)}           & O(N + E)                             & O(N²)                              & O(E)                             \\ \hline
    \textbf{Tiempo (Construcción)}            & O(N + E)                             & O(E)                               & O(E)                             \\ \hline
    \textbf{Acceder a los Vecinos (por Nodo)} & O(1)                                 & O(N)                               & O(E)                             \\ \hline
    \textbf{Tiempo para Buscar Camino}       & O(N + E) (DFS/BFS)                   & O(N²)                              & O(E * k)                         \\ \hline
    \textbf{Espacio Temporal (Búsqueda)}      & O(N)                                 & O(N)                               & O(N)                             \\ \hline
    \textbf{Acceder a la existencia de una Arista} & O(k) (conjunto de vecinos)    & O(1)                               & O(E)                             \\ \hline
    \end{tabular}
    \caption{Comparativa de complejidad temporal y espacial de las representaciones de grafos.}
    \end{table}
\end{document} 

